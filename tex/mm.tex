% This file is part of the Moving Mesh project.
% Copyright 2017 the authors. All rights reserved.

\documentclass[modern]{aastex61}

\usepackage{url}
% \pagestyle{plain}
% \usepackage[pass,letterpaper]{geometry}

% typography
\setlength{\parindent}{1.\baselineskip}
% \newcommand{\acronym}[1]{{\small{#1}}}
% \newcommand{\CRLB}{\acronym{CRLB}}

% aastex parameters
%%\hypersetup{linkcolor=red,citecolor=green,filecolor=cyan,urlcolor=magenta}
% \received{not yet; THIS IS A DRAFT}
%\revised{not yet}
%\accepted{not yet}
%% Adds "Submitted to " the arguement.
%\submitjournal{ApJ}
\shorttitle{Kinematics of cold stellar streams}
\shortauthors{Bonaca et al.}

\begin{document}
\sloppy\sloppypar\raggedbottom\frenchspacing

\title{A Moving Mesh: Kinematics of cold stellar streams in the Milky Way}

\correspondingauthor{Ana Bonaca}
\email{ana.bonaca@cfa.harvard.edu}

\author[0000-0002-7846-9787]{Ana Bonaca}
\affil{Harvard--Smithsonian Center for Astrophysics}

\author{coauthors}

\begin{abstract}
\noindent
Large-scale photometric surveys have revealed that the halo of the Milky Way is crisscrossed with cold stellar streams.
These structures are faint and diffuse, so out of several dozen of discovered streams, only a handful have been followed up spectroscopically.
Within the Moving Mesh project, we are employing large field-of-view spectroscopes on 6.5\,m telescopes to kinematically characterize cold streams in the Milky Way.
This is the first in a series of Moving Mesh papers, where we targeted the disrupting globular cluster NGC~5466 and a distant stellar stream Styx, and provide kinematic information of their tidal debris for the first time.
\end{abstract}

% \keywords{}


\section{Introduction}
\label{sec:intro}
- mesh of streams: summary of observations

- kinematics important to confirm
- even more so for modeling
- basic halo properties still uncertain
- most of these streams are in the inner halo, so not much they can tell about the total mass
- but should be very constraining on the shape
- theoretical motivation: lcdm predicts complex shape, changes with radius, under influence of baryons -- more spherical in the inner parts
- halos with other dm candidates have unique shape predictions -- we can test that (e.g., axion -- wiggles)
-- need velocities to do that, that's why we're launching MM

Tidal tails around NGC~5466 were discovered by \citet{belokurov2006} and \citet{gj2006}.
- explored for halo shape -- should be very constraining joint with pal5

- styx reported in grillmair2009
- also nearby Bootes III overdensity, hypothesized as a progenitor
- radial velocities obtained for BooIII


\section{Data}
\label{sec:data}
We observed the tidal tails of NGC~5466 and Styx stellar streams with MMT/Hectochelle \citep{hectospec} in February and May 2017.
The observations were taken using the 110\,$l$/mm grating and the `RV31' order-blocking filter over the wavelength range 5150--5300\,\AA, with exposure times of 1.5\,hours per field.
A total of three NGC~5466 and four Styx fields were observed.
The basic image reduction was done by the TDC \citep{hectoredux}, and the radial velocities were measured following \citet{caldwell2017}.
Out of 739 targets along the NGC~5466 tidal tails, we were able to obtain radial velocities for 251 stars.
For Styx, the success rate was higher, and radial velocity measurements were made for 604 out of 867 targeted stars.
Median velocity uncertainty is $\lesssim1\;$km/s for both streams.
Our preliminary results are summarized in Figures~\ref{fig:summary_ngc5466} and \ref{fig:summary_styx} for NGC~5466 and Styx, respectively.


\section{Kinematics}
\label{sec:vr}
- both streams produce and overdensity on the expected distribution for a smooth Milky Way
- next step: logg to remove dwarfs
- membership probability: logg, feh?
- table with radial velocities

\subsection{NGC 5466}
\label{sec:ngc5466}

\begin{figure}
\begin{center}
\includegraphics[width=\textwidth]{../plots/summary_ngc5466.pdf}
\caption{Summary of NGC~5466 observations.
(\emph{Left}) Velocity distribution of MMT/Hectochelle targets with well-measured radial velocities (empty black histogram).
The expected distribution for a smooth halo from the Besan\c{c}on model is shown as an orange-filled histogram.
We define kinematic membership to the NGC~5466 stream for stars with radial velocities between $\sim80-140\;$km/s, where the Milky Way measurements show a clear overabundance, and highlight this range in gray.
(\emph{Center}) Color-magnitude diagram of targeted stars (gray) and kinematic members (black).
Targets were selected using an old and metal-poor isochrone at a distance of 16.2\,kpc.
Several of the potential member stars are on the BHB, while the rest are at the main sequence turn-off ($r\sim19$), with very few members along the RGB.
(\emph{Right}) Radial velocity of NGC~5466 members as a function of R.A.
Close to the cluster ($\rm R.A.=211.364^\circ$), there is no discernible gradient in radial velocity, but potential members in the most distant field, $\rm R.A.\approx202.5^\circ$, have a large velocity spread.}
\label{fig:summary_ngc5466}
\end{center}
\end{figure}

\subsection{Styx}
\label{sec:styx}

\begin{figure}
\begin{center}
\includegraphics[width=\textwidth]{../plots/summary_styx.pdf}
\caption{Summary of Styx observations.
(\emph{Left}) Velocity distribution of MMT/Hectochelle targets with well-measured radial velocities (empty black histogram).
The expected distribution for a smooth halo from the Besan\c{c}on model is shown as an orange-filled histogram.
We define kinematic membership to the Styx stream for stars with radial velocities between $-240\;$km/s and $-120\;$km/s, where the Milky Way measurements show a clear overabundance, and highlight this range in gray.
(\emph{Center}) Color-magnitude diagram of targeted stars (gray) and kinematic members (black).
Targets were selected using an old and metal-poor isochrone at a distance of 48\,kpc.
Some of the kinematic members are potentially BHB stars ($g-r\lesssim0.4$), while most are found along the RGB.
(\emph{Right}) Radial velocity of Styx members as a function of R.A.
The contamination from the Milky Way foreground among the potential kinematic members is on the order of $\sim50\%$, so determining the radial velocity along the stream more precisely would benefit from measuring surface gravities and removing dwarf stars from this sample.}
\label{fig:summary_styx}
\end{center}
\end{figure}

\section{Dynamics}
\label{sec:dynamics}
- inner halo constrained by pal5 and gd-1 to be spherical (bovy)
- jeans modeling (sarah loebman) -- oblate, angus williams? (or some other cambridge student) -- prolate
- lux showed that kinematics of ngc 5466 tails should be a good discriminator of shape
- first assume halo scale (mass and radius) from bovy (also matches disk kinematics), then take measured 6D position of the cluster, and create models assuming different flattening
- contrast the predicted radial velocities to the measured one
- fit for it?

- outer halo: styx important because far out
- can test if there is radial change in halo shape, or styx radial velocities consistent with shape of the inner halo

\section{Discussion}
\label{sec:discussion}

\section{Summary}
\label{sec:summary}

\vspace{0.5cm}
\emph{Acknowledgments:}
This paper uses data products produced by the OIR Telescope Data Center, supported by the Smithsonian Astrophysical Observatory.

\bibliography{mm}

\end{document}
